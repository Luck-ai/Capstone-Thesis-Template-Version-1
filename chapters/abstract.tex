
\noindent
\begin{tabular}{ll}
\textbf{Thesis} & Thesis Title in English \\
\textbf{Student} & Mr. Name Surname \\
\textbf{Student ID.} & XXXXXXXX \\
\textbf{Degree} & Bachelor of Engineering \\
\textbf{Program} & Robotics and AI Engineering \\
\textbf{Year} & 20XX \\
\textbf{Thesis Advisor} & Prof. Dr. Thesis Advisor \\
\textbf{Co-Thesis Advisor} & Assoc. Prof. Dr. Co-Thesis Advisor \\
\end{tabular}

\vspace{1cm}

\begin{center}
\textbf{\Large ABSTRACT IN ENGLISH}
\end{center}

\vspace{0cm}

A good abstract writing should focus only on the key and interesting aspects of
the research. It should emphasize the specific highlights of the research, with the
research work being clear, concise, and meeting certain criteria for a good abstract. For
example, the word count should be between 200-250 words, or approximately no
more than half of an A4 page. Before writing the abstract, one should read and
understand their own complete research work to identify various interesting points
that can engage general readers, making them want to read the full research. There
should be no interpretation or criticism using personal opinions. Complex or aggressive
language should be avoided. Local jargon, unnecessary abbreviations, or symbols
should not be used as they may lead to misunderstanding. There should be no
numerical references, diagrams, tables, various statistical formulas, or equations in the
abstract, unless necessary to display analytical results.
The abstract may have multiple paragraphs to enhance reader comprehension
in each section. Avoid referencing the research work of others in the abstract. The
structure of the abstract should also be carefully considered, including font type, font
size, page margin, printing standards, and an acceptable format. Proofreading the
writing by reading it multiple times will improve the quality of the abstract, making it
acceptable for research presentations.
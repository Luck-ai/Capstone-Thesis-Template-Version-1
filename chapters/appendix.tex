\appendix

\chapter{GUIDELINES AND PROCEDURES FOR THESIS WRITING}
\thispagestyle{plain}
\clearpage

\chapter*{GUIDELINES AND PROCEDURES FOR THESIS WRITING}
\addcontentsline{toc}{chapter}{GUIDELINES AND PROCEDURES FOR THESIS WRITING}

Thesis writing is considered a crucial part for graduate students after completing coursework at a certain level. Students must plan their thesis writing process until the comprehensive thesis examination is completed.

\section*{Procedures for Requesting Approval of Thesis Topics and Scheduling Thesis Examinations:}

\begin{enumerate}
    \item Graduate students have the right to propose thesis topics as soon as they become enrolled in the graduate program.
    
    \item Students must submit a request for approval of the thesis topic to the graduate school of their respective faculty.
    
    \item The approval of the thesis topic and the thesis outline is within the authority of the dean, endorsed by the academic affairs committee.
    
    \item Students must receive approval of the thesis topic and the outline (if any) before the final thesis examination.
    
    \item Any changes before the thesis examination must be handled as follows:
    \begin{enumerate}
        \item \textbf{Change of thesis topic or outline:} Students must submit a request to the dean, with approval from the thesis advisor and endorsement from the academic affairs committee.
        
        \item \textbf{Change of thesis advisor:} Students must submit a request to the dean of the respective academic unit, and the dean has the authority to approve, with endorsement from the academic affairs committee.
        
        \item In cases where the thesis advisor loses their position within the academic unit, the student must find a new advisor, and the work specified by the owner of the work, including the student's name, the name of the thesis advisor, the name of the academic unit, and the name of the institution, should be used to support the thesis examination application.
    \end{enumerate}
    
    \item The thesis examination is the final stage, and students must comply with the conditions and requirements of each program in their respective academic units. Students can only take the thesis examination when:
    \begin{enumerate}
        \item The thesis advisor believes that the student is ready for the examination. The student must notify the examination committee and submit a draft of the thesis to the specified number of graduate students, faculty, and the examination committee no less than two weeks in advance.
        
        \item For master's degree examinations, the branch chairman and the course coordinator must conduct the thesis examination committee, endorsed by the academic affairs committee. The committee should consist of five members, including the thesis advisor, one graduate faculty member proposed by the thesis advisor, and three members, including an outside faculty member who is an expert, proposed by the academic affairs committee. The committee must appoint the external committee member as the committee chairman.
        
        \item For doctoral degree examinations, the dean should appoint the thesis examination committee, endorsed by the academic affairs committee. The committee should consist of five members, including the thesis advisor, one faculty member proposed by the thesis advisor, and three members, including an outside faculty member who is an expert, proposed by the academic affairs committee. The committee must appoint the external committee member as the committee chairman.
    \end{enumerate}
\end{enumerate}
\chapter{INTRODUCTION}

\section{SIGNIFICANCE OF A THESIS}
A thesis is a continuous presentation of a study or research undertaken by
students or researchers. Research or study at the graduate level differs significantly
from undergraduate study because it involves in-depth exploration, problem
identification, and a well-defined problem-solving process. It presents reasoned
theories, analysis, and logically sound critiques from various perspectives. Therefore, a
thesis is a written work or literature that must be studied, referenced, or further
researched by subsequent students or researchers. A good thesis should have the following characteristics:
\begin{enumerate}[label=\arabic*.]
    \item It should address a specific research question or problem.
    \item It should demonstrate a thorough understanding of the relevant literature.
    \item It should employ appropriate research methodologies.
    \item It should present original findings or contributions to the field.
    \item It should be well-organized and clearly written.
\end{enumerate}

\section{THE IMPORTANCE OF THE THESIS PRESENTATION PROCESS}
The sequence of steps in presenting or writing a thesis is of paramount
importance for the understanding of the readers, whether they are the thesis
examiners themselves or individuals who will use the thesis for research or reference
in subsequent ranks. The central content of the thesis should ideally consist of three
main subsections: the introduction, the main content section, and the conclusion.

\section{Scope and Limitations of the Project}
...
\section{Methodology Used in the Project}
...
\section{Project Implementation Plan}
...
\section{Benefits Derived from the Research}
...
\section{Capstone Design Details}
\subsection{APPLICATION OF KNOWLEDGE AND SKILLS FROM PREVIOUS COURSES}
\subsection{DESIGN OF A SYSTEM, PROCESS, OR COMPONENT/DEVICE}
\subsection{COMPLIANCE WITH INDUSTRIAL STANDARDS}
\subsection{DESIGN CONSTRAINTS}
\subsection{TEAMWORK, REPORT PREPARATION, AND ORAL PRESENTATION}
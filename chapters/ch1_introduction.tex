\chapter{INTRODUCTION}

\section{SIGNIFICANCE OF A THESIS}

A thesis is a continuous presentation of a study or research undertaken by
students or researchers. Research or study at the graduate level differs significantly
from undergraduate study because it involves in-depth exploration, problem
identification, and a well-defined problem-solving process. It presents reasoned
theories, analysis, and logically sound critiques from various perspectives. Therefore, a
thesis is a written work or literature that must be studied, referenced, or further
researched by subsequent students or researchers. A good thesis should have the following characteristics:
\begin{enumerate}[label=\arabic*.]
    \item It should address a specific research question or problem.
    \item It should demonstrate a thorough understanding of the relevant literature.
    \item It should employ appropriate research methodologies.
    \item It should present original findings or contributions to the field.
    \item It should be well-organized and clearly written.
\end{enumerate}

\section{THE IMPORTANCE OF THE THESIS PRESENTATION PROCESS}

The sequence of steps in presenting or writing a thesis is of paramount
importance for the understanding of the readers, whether they are the thesis
examiners themselves or individuals who will use the thesis for research or reference
in subsequent ranks. The central content of the thesis should ideally consist of three
main subsections: the introduction, the main content section, and the conclusion.

\subsection{Introduction Section}

The introduction section serves as the first chapter of the thesis, following the
abstract. Many theses have a distinct separation between the abstract and the
introduction. The abstract provides a concise overview of what the problem is, what
the author has done, how it was done, and what the outcomes were. The introduction,
on the other hand, leads the reader into the problem systematically. It often
summarizes previous research in the field and then explicitly identifies the clear
problem (problem identification) that the author will address or solve, followed by
outlining the study or problem-solving process. This leads to the transition into the
second section, which is the main content section.

\subsection{Main Content Section}

The main content section of the thesis is the largest part of the entire
document. It typically comprises several chapters, ranging from 2 to 5 chapters or
more, with each chapter having a similar length and content. Each chapter should not
be overly long to ensure readability (usually around 20-40 pages). The first chapter in
this section often discusses general principles or relevant theories and the research
conducted by others (literature review). Subsequent chapters detail various stages of
the study or problem-solving process, along with the results obtained. Many theses in
this section conclude each chapter and provide an introduction to the next chapter.

\subsection{Conclusion Section}

This section of the thesis is crucial and consists of no fewer than the first two
sections. It demonstrates the achievement of the study's objectives and showcases
the depth of the researcher's understanding. It involves presenting analysis, critique, or
recommendations. The conclusion section and the summary section should be clearly
distinguished because the conclusion typically provides a concise overview of the
study or research results.

\section{LANGUAGE}

In thesis writing, the author can write in either Thai or English. The language
used for presenting the thesis is written language, not spoken language, and it is not
about quantity but quality. Therefore, each page of the thesis must be concise, clear,
and adhere to the principles of correct vocabulary and grammar usage. The use of
technical terminology or foreign language words should also be considered. If possible,
use translated terms or standard vocabulary in the Thai language (if the thesis is written
in Thai). If unsure about conveying the meaning correctly, include the foreign language
term in parentheses.
It is essential to understand that the language used in thesis writing is a means
of mutual understanding between the author and the readers of that thesis. In addition
to using the correct words, the order of words presented is equally important. Each
paragraph should clearly indicate its main point, and the transitions between adjacent
paragraphs should serve as a good bridge for readers to follow seamlessly.
Redundancy, misleading statements, and ambiguity should be avoided in thesis
writing. Typically, authors do not intend for these issues to arise, but they often occur
due to lack of carefulness. Some sentences or paragraphs may appear non-repetitive
at first glance but summarizing them may reveal that they reiterate the same points
made previously. Such repetition can lead to confusion. Always keep in mind that a
paragraph or sentence should be summarizable, and readers should not conclude that
the author is saying something different than what was previously stated.
This thesis guide provides details on various aspects of thesis writing that
authors should be aware of and adhere to the institute's regulations. In Chapter 1, the
importance of the thesis, its different parts, and the language used in thesis writing are
discussed. Chapter 2 provides details and formats for various parts of the thesis, starting
from the cover page. Chapter 3 elaborates on the format of paper size, printing
standards, and an acceptable format. The methods of referencing and writing
bibliography and citations are discussed in Chapters 4 and 5, respectively. Appendices
are divided into three main sections: the process of proposing a thesis and requesting
thesis examination, various examples of thesis writing, the institute's regulations
regarding graduate studies, and related form templates.

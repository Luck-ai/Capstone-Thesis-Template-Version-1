\chapter{THESIS ORGANIZATION AND COMPONENTS}

\section{THESIS COMPONENTS - PART 1: PRELIMINARY SECTION}

This section includes:
\begin{enumerate}
    \item COVER PAGE
    \item ENGLISH TITLE PAGE
    \item COPYRIGHT PAGE
    \item APPROVAL SHEET
    \item THAI ABSTRACT
    \item ENGLISH ABSTRACT
    \item ACKNOWLEDGEMENT
    \item TABLE OF CONTENTS
    \item LIST OF TABLES (if any)
    \item LIST OF ILLUSTRATION OR FIGURES (if any)
\end{enumerate}

\section{THESIS COMPONENTS - PART 2: MAIN SECTION}

This section includes:
\begin{enumerate}
    \item Introduction
    \item Main Content, which may consist of:
    \begin{enumerate}[label=2.\arabic*]
        \item Literature Review (or related research) (LITERATURE REVIEW)
        \item Research Methodology (RESEARCH METHODOLOGY), which may have multiple chapters
        \item Experimental Results or Data Analysis (RESULTS OR ANALYSIS OF DATA)
        \item Critique or Discussion of Findings (DISCUSSION)
    \end{enumerate}
    \item Conclusion, which may include:
    \begin{enumerate}[label=3.\arabic*]
        \item Research Summary and Recommendations (CONCLUSION AND SUGGESTION)
    \end{enumerate}
\end{enumerate}

\section{THESIS COMPONENTS - PART 3: FINAL SECTION}

This section includes:
\begin{enumerate}
    \item BIBLIOGRAPHY OR REFERENCES (if any)
    \item APPENDIX, APPENDICES (if any)
    \item AUTHOR BIOGRAPHY
\end{enumerate}

\section{EXPLANATION OF THESIS COMPONENTS}

\subsection{Part 1: Preliminary Section}

\begin{enumerate}[label=\arabic*)]
    \item COVER

The top section includes the thesis title in English, separated by one line.

The middle section includes the author's name in English, without using any prefixes such as Mr., Mrs., Miss, etc.

The bottom section includes the statement:
\begin{center}
\textbf{A THESIS SUBMITTED IN PARTIAL FULFILLMENT}\\
\textbf{OF THE REQUIREMENT FOR THE DEGREE OF\ldots}\\
\textbf{OF ENGINEERING IN \ldots}\\
\textbf{SCHOOL OF ENGINEERING}\\
\textbf{KING MONGKUT'S INSTITUTE OF TECHNOLOGY LADKRABANG}\\
\textbf{20XX (year of thesis submission)}\\
\textbf{KMITL-20XX-EN-X-XXX-XXX}
\end{center}

\textbf{\underline{Note}}
\begin{enumerate}[label=\alph*)]
    \item For study programs that require coursework (Plan 1.2), use the phrase:
    \begin{center}
    \textbf{A THESIS SUBMITTED IN PARTIAL FULFILLMENT}\\
    \textbf{OF THE REQUIREMENT FOR THE DEGREE OF\ldots}
    \end{center}
    
    \item For study programs that do not require coursework but focus solely on research (Plan 1.1), use the phrase:
    \begin{center}
    \textbf{A THESIS SUBMITTED IN FULFILLMENT}\\
    \textbf{OF THE REQUIREMENT FOR THE DEGREE OF\ldots}
    \end{center}
\end{enumerate}

    \item ENGLISH TITLE PAGE The content is the same as the cover in all respects.

    \item COPYRIGHT PAGE Print on the left-bottom margin of the page in English.

    \item APPROVAL SHEET The graduate affair issues the approval sheet. The Dean of the School of Engineering signs the approval sheet upon successful defense.

    \item THAI ABSTRACT Include the thesis title, student's name, student ID, degree name, department name, year of thesis publication, thesis advisor's name, and co-advisor's name (if applicable).

    \item ENGLISH ABSTRACT Follow the same format and content as the Thai abstract.


    \textbf{\underline{Note:}} The thesis should be written in both Thai and English, and both versions should have abstracts.

\item ACKNOWLEDGEMENT Express gratitude to individuals who contributed to the successful completion of the thesis, including the thesis advisor and sources of funding (if applicable).

\item TABLE OF CONTENTS List the page numbers in Roman numerals (I, II, III, IV...) for the abstract, introduction, and all the way to the last page.

\item LIST OF TABLES List the page numbers for tables in sequential order, including tables in the appendices.

\item LIST OF ILLUSTRATIONS OR FIGURES List the page numbers for illustrations, maps, graphs, etc., in sequential order, including those in the thesis appendices.

\end{enumerate}



\subsection{Part 2: Main Section}

\subsubsection{Introduction}

The introduction, which is Chapter 1, serves as the beginning of the main content. It may include the following:

\begin{enumerate}[label=\arabic*)]
    \item STATEMENT AND SIGNIFICANCE OF THE PROBLEMS

    Discuss the background and significance of the research topic. Explain why this topic is being investigated, what problems are of interest in the research, what benefits the research can provide, and how it will contribute to the overall understanding.

    \item GOAL AND OBJECTIVE OF THE STUDY

    Clearly state the goal and objectives of the study, specifying what the study aims to prove or discover.

    \item HYPOTHESIS

    Present the hypotheses that will be tested in the study, based on the research objectives and referencing relevant principles and theories.

    \item SCOPE OF THE STUDY

    Define the scope of the study by specifying what the study will cover and how broad or limited it is.

    \item PROCESS OF THE STUDY

    Summarize the steps or processes involved in conducting the study.

    \item ASSUMPTIONS

    State any preliminary assumptions or conditions that apply to the study.

    \item LIMITATION OF THE STUDY

    Identify any constraints or variables that cannot be controlled, such as time limitations, budget constraints, or other factors.

    \item DEFINITIONS

    Provide definitions for terms or concepts that are specific to the study. Technical terms and abbreviations may be compiled in a list and included in or after the table of contents. This chapter sets the stage for the research, providing readers with a clear understanding of the background, goals, and parameters of the study.
\end{enumerate}

\subsubsection{Main Content}
\begin{enumerate}[label=\arabic*)]
    \item Literature Review

    The literature review discusses existing theories, ideas, literature, or research related to the study. It serves as a guide to the research process, highlighting previous work that is relevant and significant to the current study.

    \item Research Methodology

    The research methodology section provides details on the following:
    \begin{itemize}
        \item Research Methods: Specify the research methods employed, such as document analysis, surveys, or experiments.
        \item Data Characteristics and Selection: Explain the nature of the data used and the rationale for selecting specific data sources.
        \item Tools and Techniques: Describe the tools and techniques used in data collection and analysis.
        \item Data Collection Procedures: Outline the steps taken to gather data.
        \item Data Analysis Methods: Explain the methods used to analyze the data, which may include statistical techniques.
    \end{itemize}

    \item Results or Analysis of Data

    This section presents the detailed results of the study, including tables and figures if applicable. Statistical analysis may be used to provide clear interpretations of the data.

    \item Discussion

    The discussion section serves several purposes:
    \begin{itemize}
        \item To highlight the underlying principles demonstrated by the study.
        \item To support or challenge previously proposed theories.
        \item To compare the study's results with those of others.
        \item To summarize key findings and implications of the research.
    \end{itemize}

    The author should emphasize critical issues or controversies related to the main content and offer recommendations for future research.

    \item Tables and Figures (if applicable)

    Tables and figures should be integrated into each chapter of the main body where relevant.
\end{enumerate}

\subsubsection{Conclusion}

The conclusion and suggestions section is the final chapter. It summarizes the research findings, starting from the research inception through the completion of the study. It highlights the significance of the work, the key results obtained, and provides an analysis of the overall research process.

Suggestions refer to recommendations or insights that can enhance the research, focusing on areas for improvement or further analysis within the research work. It aims to suggest ways to enhance efficiency, accuracy, or comprehensiveness to maximize the value and utility of the research for future users and researchers.

\subsection{Part 3: Final Section}

\subsubsection{Reference List}

In the concluding section of your thesis, you should provide a reference list that includes all the sources you cited in your thesis. You have the option to use either a bibliography or a references section, but you should stick to one method consistently throughout your thesis.

Bibliography: This is a list of books, documents, or other media that you consulted while conducting your research, presented in a name-year format within the body of your thesis.

References: This is a list of sources cited in your thesis, presented in a numbered format, typically following a specific citation style like APA, MLA, Chicago, or any other style relevant to your field.

\subsubsection{Appendices (if applicable)}

The appendices section is meant to provide additional information that can help readers gain a more detailed understanding of your thesis content. It may include items such as interviews, questionnaires, research timeframes, budget details, related research publications, or any other relevant data. You can have multiple appendices if needed, and they should be labeled as Appendix A, Appendix B, Appendix C, and so on.

\subsubsection{Author Biography}

The author biography section provides information about the author's background and qualifications. It typically includes the author's full name, date and place of birth, educational qualifications (including degrees, institutions, and graduation years), academic achievements, awards, scholarships, current affiliation, work experience, and current position or role. This section helps readers understand the author's expertise and credibility in the subject matter of the thesis.
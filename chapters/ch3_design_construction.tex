\chapter{PAGE LAYOUT}

The students must study the requirements and formatting for their theses according to the thesis guidelines of the graduate studies in order to ensure that the paper outline meets the standard. Prior to the final thesis defense presentation, students are required to submit the original version to their thesis advisor for review to ensure accuracy and uniformity.

\section{PAPER SIZE}

The paper used for printing the thesis content must be white, unlined, A4 size paper (210 mm wide and 297 mm long).

\section{FONT}

For the thesis cover, use 20-point font size for the title in Thai (approximately 3 mm in height) and 18-point font size for the title in English (approximately 2.5 mm in height). Use the TH Sarabun New font style. The font size for the inside cover should be the same as the outside cover.

The font size for the chapter titles, section headings, and content can be found in the following section.

\section{LINE SPACING}

Use a 7-character word space, which is approximately 60 characters per line. Please indent the first line of each paragraph.

\section{MARGINS}

The top margin should be 1 inch (25.4 mm) from the paper's edge.

The left margin should be 1.5 inches (38.1 mm) from the paper's edge.

The right margin should be 1 inch (25.4 mm) from the paper's edge.

The bottom margin should be 1 inch (25.4 mm) from the paper's edge.

\section{PAGE NUMBERING}

For Part 1, starting from the abstract to the bibliography (if present), use Roman numerals (I, II, III, IV, V) to indicate page numbers. Print them at the bottom center of each page.

For the main content section, use Arabic numerals 1, 2, 3, 4, 5 to indicate page numbers. Print them at the top right, 0.5 inches from the top edge of the paper, and 1 inch from the outer edge of the paper.

The first page of each chapter does not need a page number, but the pages should be counted.

\section{CHAPTER DIVISION, MAIN HEADINGS, AND SUBHEADINGS}

Chapter numbers, such as Chapter 4, should be centered on the page and in bold with a 20-point font size.

The chapter title should be printed in the center of the page, in bold with a 24-point font size. No chapter numbering is required.

Major headings, which are titles other than the chapter title, should be aligned to the left margin, followed by a chapter number and a period (.), and then the heading number. Use bold 18-point font size and leave a 1-line space above.

Subheadings, which divide the major headings, should be indented from the left margin by 7 spaces. Use the number of the major heading, followed by a period (.), and then the subheading number. Leave a 2-space gap and then use a bold 16-point font size. Leave a 1/2-line space above.

\noindent
\begin{flushleft}
\hspace{7em}1.1 Major Heading \dotfill \\
\hspace{8em}1.1.1 Sub heading of 1.1 \dotfill \\
\hspace{9em}1.1.1.1 Sub Heading of 1.1.1 \dotfill \\
\hspace{10em}1) Sub Heading of 1.1.1.1 \dotfill
\end{flushleft}

For any section headings, capitalize the first letter of each word. The first letter of each word in major headings and subheadings must be capitalized.

Use black Thai Sarabun New font style with a 16-point font size (approximately 2 mm in height) for the content text. Maintain the same font style throughout the document for symbols or characters that the printer does not provide. Use black ink, preferably of high quality.

\section{TABLES}

Insert tables within the relevant sections of the main text, leaving one line space before printing the word "Table" followed by the table number in bold, aligned to the left margin. Then, print the table name. If the table name is longer than one line, print the top line longer than the bottom line, starting in line with the first letter of the table name. Continue the table without leaving a line if needed. If a table is too wide to fit on one page and must be continued on the next page, align it to the left margin and print the word "(continued)" followed by the table number, like "Table 3.1". After finishing the table, leave one line space before the next regular text.

(Leave one line space)

\textbf{Table 3.1} Example of Table

\begin{table}[h]
\centering
\begin{tabular}{|c|c|c|c|c|c|c|}
\hline
\multicolumn{2}{|c|}{Spacing} & \multicolumn{5}{c|}{Modulus \%} \\
\hline
S$_1$ (cm) & S$_2$ (cm) & 0 & 30 & 50 & 80 & 100 \\
\hline
3.3 & 7.5 & 49 cm$^2$ & 64 cm$^2$ & 81 cm$^2$ & 85 cm$^2$ & 90 cm$^2$ \\
3.8 & 8.5 & 60 cm$^2$ & 75 cm$^2$ & 90 cm$^2$ & 97 cm$^2$ & 100 cm$^2$ \\
4.3 & 9.5 & 70 cm$^2$ & 81 cm$^2$ & 95 cm$^2$ & 99 cm$^2$ & 109 cm$^2$ \\
4.8 & 10.5 & - & - & - & - & - \\
\hline
\end{tabular}
\end{table}

(Leave one line space)

\section{IMAGES}

To insert images, leave one line space before placing the image in the center of the page. Include the words "Picture." or "Figure" (use either one consistently throughout the document), followed by the image number, using bold font. The image description should not exceed one line and should be placed in the center beneath the image. If the description extends beyond one line, place it on the left edge. Number the images or pictures in a manner similar to how tables are numbered, following the end of the image description. Leave one line space before continuing with regular text.

(Leave one line space)

\begin{figure}[h]
\centering
% Placeholder for actual figure content
\rule{6cm}{4cm} % This represents a placeholder for an actual figure
\caption{Example of Figure}
\label{fig:3.1}
\end{figure}

(Leave one line space)

\section{TABLE OF CONTENTS, TABLE OF TABLES, TABLE OF FIGURES (OR LIST OF FIGURES)}

Print the words "Table of Contents," "Table of Tables," "Table of Figures" (or "List of Figures") in the center of the page, 1 inch from the top margin, using 24-point bold font.

Leave one line space and then print the word "Page" aligned to the right. Print the page numbers flush with the right margin and connect them to the content with dots.

Next, after one line space, will be the content of the table of contents, between different chapters, bibliography, and appendices. Leave one line space. As for the Table of Tables and Table of Figures (or List of Figures), print the words "Table \#" or "Figure \#" aligned to the left margin, on the same line as "Page."

\section{ABBREVIATION LIST OR SYMBOL LIST}

Use this for cases when the abbreviation list is printed separately from the introduction and follows the instructions in section 4.10. Print the words "Abbreviation List" or "Symbol List" (or "Abbreviation and Symbol List") in the center of the page, 1 inch from the top margin, using 24-point bold font.

Leave one line space, and then start printing the abbreviations or symbols flush with the left margin. If the description of the abbreviation or symbol doesn't fit on one line, start the next line from the 8th character, as before.

For English letters, Greek letters, and symbols with upper/lower marks and numeric marks: English Letters: A, B, C, b, c Greek Letters: C, P, Y,...

If there are English letters mixed with Greek letters, the Greek letters come after the English letters. Symbols with Upper/Lower Marks and Numeric Marks: A$_1$ A$_x$ A$^1$ A$^x$ A$_1^x$ a$_1$ a$_x$ a$^1$ a$^x$ a$_1^x$

In this order, symbols with lower marks come before symbols with upper marks. Additionally, symbols with numeric marks come before symbols with alphabetic marks.

\section{SCIENTIFIC NAMES}

For biological organisms, plants, and animals, use the International Code of Nomenclature for emphasizing scientific names. Make them stand out distinctly from other text by underlining or using italics. Scientific names follow the binomial system, consisting of two parts. The first part is the Genus name, which begins with an uppercase letter, and the second part is the Specific epithet, written slightly offset and beginning with a lowercase letter. In scientific names, it's common to include the name of the person who initially defined the name and described the organism. Personal names are typically used for surnames only. If it's a well-known name, it may be abbreviated, such as Linnaeus abbreviated as "Linn." or "L." In some cases, when two individuals co-author a name, both names are used, for example...

\begin{flushleft}
\begin{tabular}{@{}l l}
\hspace{1in}a. micro-organism &
\begin{tabular}[t]{@{}l@{}}
    Escherichia coli, \\
    Bacillus subtilis, \\
    Azospirillum brasilense
\end{tabular} \\[1em]

\hspace{1in}b. plant &
\begin{tabular}[t]{@{}l@{}}
    Coccinia grandis L., \\
    Canna indica Linn., \\
    Cocos nucifera Linn.
\end{tabular} \\[1em]

\hspace{1in}c. animal &
\begin{tabular}[t]{@{}l@{}}
    Ptilolaemus tickeli, \\
    Panthera tigris
\end{tabular} \\
\end{tabular}
\end{flushleft}

\section{MATHEMATICAL EQUATIONS}

Mathematical equations can be inserted into the text. To maintain order, separate equations on their own lines. Equations should be positioned with a one-line space above and below. Equations should be centered on the page as appropriate and use the Cambria Math font with a font size of 11 points.

Enclose equation numbers in brackets, and order equation numbers in the same way as tables and images, for example...

\begin{equation}
S.D. = \sqrt{\frac{\sum(x - \bar{x})^2}{N - 1}}
\label{eq:3.1}
\end{equation}

\section{FOREIGN LANGUAGES}

For words in foreign languages, print a translation in English within parentheses following the necessary terms. For example, Technique (เทคนิค). Printing foreign languages without diacritical marks is common. For instance, "เทคโนโลยี" should be printed as "Technology" Compound words should not use "ส" or "ส์" in Thai. For example, "Games" should be printed as "เกม," except for proper names like "SEAGAMES," which should be printed as "ซีเกมส์."

\section{ABBREVIATIONS}

b\&w (black and white) means black and white images.

c. (copyright) means used for copyright year.

ca. (circa) means an approximate date.

ch. (chapter) means chapters in royal decrees, laws, etc.

chap. (chapter) means regular chapters; use "chaps." for plural.

col. (color) means color photographs.

ed. (edition ; editor ; edited by) means editor or edited by.

enl. (enlarged) means an enlarged edition, e.g., "enl. ed.".

et al. (et alii) means and others.

fig. (figure) means figures, plural "figs.".

fr. (frame) means frame in slides to indicate the number of frames.

i.p.s. (inches per second) means inches per second in tape speed.

ill. (illustrated by) means illustrated by.

min. (minutes) means film duration in minutes.

ms. (manuscript) means manuscript, plural "mss.".

n.p. (no place; no place of publishing) means no place of publication.

no. (number) means number, plural "nos.".

2nd. ed. (second edition) means the second edition.

p. (page) means page, plural "pp.".

Par. (paragraph) means paragraph, plural "pars.".

Pt. (part) means part, plural "pts.".

r.p.m. (revolutions per minute) means revolutions per minute.

rev. (revised) means revised edition, e.g., "rev. ed.".

3rd ed. (third edition) means the third edition.

sc. (scene) means scene.

sd. (sound) means sound in films.

sec. (section) means section, plural "secs.".

Si. (silent) means silent films.

trans. (translator; translated by) means "translator" or "translated by".

Vol. (volume) means volume, plural "vols.".

\section{PUNCTUATION MARKS}

Period (.) should be printed with a 2-space gap between words.

Comma (,) should be printed with a 1-space gap between words.

Semicolon (;) should be printed with a 1-space gap between words.

Colon (:) should be printed with a 1-space gap between words.

Quotation marks (" ") should be printed with a 1-space gap between words.

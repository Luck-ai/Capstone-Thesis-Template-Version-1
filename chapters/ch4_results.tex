\chapter{REFERENCE AND CITATION}

Citation refers to the reference documents used in thesis writing. It is a way to inform the sources of the text, ideas, or any content that is not originally created by the author. This is done as a tribute to individuals or organizations who own the intellectual property of the ideas or the data. It also serves the convenience of readers who wish to know more details from the original source to be able to trace accurately.

Citations can either summarize the original content or can be a direct quotation, maintaining the original writing format. To ensure accuracy, all information in the reference list must be precise and thoroughly checked for their sources.

Citations come in two main formats:
\begin{itemize}
    \item In-text citation using the name-year system.
    \item In-text citation using the numerical sequence system.
\end{itemize}

At the end of the thesis, all the documents and references cited can be organized alphabetically according to the authors' names in a bibliography or in numerical order according to the reference numbers in a reference list.

\section{IN-TEXT CITATION IN THE NAME-YEAR SYSTEM}

Citations should include the author's name and the publication year, with the option to include page numbers separated by a colon (:).

Author names in Thai documents should include both first and last names, while in foreign language documents, only the last name should be used.

The citation format may vary depending on the sentence structure, either placed at the beginning or end of a sentence.

\textbf{Example (Author mentioned at the beginning of the sentence):}

Suttilak Ampanswong (2521: 25) explains that information refers to various knowledge, news, and data.

Good (1973: 112) defines commitment as an individual's expression of love and concern.

\textbf{Example (Author mentioned at the end of the sentence):}

...managers must have the ability to manage people to foster cooperation and collaboration in responsive and well-coordinated work, leading to the efficient achievement of organizational objectives (Uthai Boonprasert, 2531: 23).

In citations, there may be different cases as follows:

\subsection{Single Author}

Mayuree Chaisawat (2538: 86)...

Heyes (1964)...

\subsection{Two Authors}

Prayad Janthachomphu and Prasopas Aksonmat (2518: 24)...

Macaulay and Berkowitz (1978: 4)...

\subsection{More than Two Authors}

Sanan Jittrasuk et al. (2532: 21-25)...

Bradley, S. et al. (1983: 23-25)...

\subsection{Authors from an Institution, Organization, Corporation, or Agency}

From the meeting of education institution managers under the Department of Vocational Education (Department of Vocational Education, 2531) on the topic "Human Resource Needs."

...and environmental pollution, termed as 'air pollution' or the occurrence of air pollution (Department of Environmental Quality Promotion, 2539: 118).

\subsection{Multiple Titles by the Same Author in the Same Year}

Sutthas Yaksan (2529a)...

Heyes (1964c)...

\subsection{Multiple Citations within the Same Topic from Multiple Sources}

...(Yawonuch Sengyont, 2525a; Supada Intranukul, 2525)...

...(Kartner, 1973; Kartner and Russel, 1975)...

\subsection{No Author Identified}

Use 'Anonymous'

Anonymous (1973)...

\subsection{Quotations Longer Than 3 Lines}

Should be separated from the main text with a one-line space above and below. The left and right margins of both sides should be indented 8 character spaces, followed by the name-year citation in parentheses, without quotation marks ("").

\textbf{Example:}

...The personnel in the organization are the heart of the organization, a vital force that propels the organization forward. As Somyasit Naweekarn stated,

(Leave 1 line blank)

        'Internal conflicts within an organization can be designed or managed in various
        ways. Conflicts can yield positive and negative results. The positive aspect is that
        conflicts lead to the exploration of better outcomes, making the organization operate
        more efficien